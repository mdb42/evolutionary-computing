\documentclass[journal,12pt,onecolumn]{IEEEtran}
\usepackage{mathtools,amssymb,amsfonts}
\usepackage{algorithmic}
\usepackage{algorithm}
\usepackage{graphicx}
\usepackage{xcolor}
\usepackage{float}
\usepackage{setspace}
\usepackage{subcaption}
\usepackage[hidelinks]{hyperref}
\usepackage{multirow}

\doublespacing

\hypersetup{
   colorlinks=true,
   linkcolor=blue,
   citecolor=black,
   urlcolor=blue
}

\usepackage{titlesec}
\titlespacing*{\section}{0pt}{12pt plus 4pt minus 2pt}{12pt plus 2pt minus 2pt}
\titlespacing*{\subsection}{0pt}{12pt plus 4pt minus 2pt}{8pt plus 2pt minus 2pt}
\titlespacing*{\subsubsection}{0pt}{12pt plus 4pt minus 2pt}{6pt plus 2pt minus 2pt}

\title{Genetic Algorithm: Bin Packing Problem}
\author{
   \IEEEauthorblockN{Matthew D. Branson} \\
   \IEEEauthorblockA{\textit{Department of Computer Science} \\
   \textit{Missouri State University}\\
   Springfield, MO \\
   branson773@live.missouristate.edu
   }
}

\date{July 5, 2025}

\begin{document}

\maketitle

\begin{abstract}
This paper presents the implementation and analysis of a...
\end{abstract}

\begin{IEEEkeywords}
Genetic Algorithms, Bin Packing Problem, Optimization, Heuristic Search
\end{IEEEkeywords}

\section{Q1: Encoding and Initialization}

Each solution (individual) is represented by a list of integers indicating bin assignments. For example: \texttt{[0, 1, 0, 2, 1, 1, 2, 1, 3, 0]}.

\begin{itemize}
    \item This means that 10 orders are assigned to bins 0 through 3. (Orders can be 10, 25, 50, and 100).
    \item You can assume the maximum number of boxes (bins) equals the number of items.
    \item Implement a function to generate a random initial population of such individuals.
\end{itemize}

\section{Q2: Function Evaluation}

\subsection{Objective Function}

Implement a function to evaluate the objective function $f$, defined as the number of unique bins used (i.e., the goal is to minimize $f$).

\subsection{Constraint Handling}

Also compute the constraint violation $g$, defined as the number of bins whose total weight exceeds 10 kg.

Examples:
\begin{itemize}
    \item If no bin exceeds the limit, $g = 0$ (feasible).
    \item If one bin exceeds the limit, $g = 1$.
    \item If all bins exceed the limit, $g = N$ (worst case).
\end{itemize}

\subsection{Evaluation Function}

Each individual will thus be evaluated using both $f$ and $g$.

\section{Q3: GA Operations}

\subsection{Tournament Selection}

\begin{itemize}
    \item For each selection, choose two individuals at random and determine a winner.
    \item Repeat until the mating pool has $N$ individuals.
    \item Then select pairs from the mating pool for crossover.
\end{itemize}

\subsection{Winner Selection Rules}

\begin{itemize}
    \item If both individuals are feasible ($g_1 = 0$, $g_2 = 0$), the one with lower $f$ wins.
    \item If one individual is feasible and the other is not, the feasible one wins.
    \item If both are infeasible, the individual with the lower $g$ wins.
\end{itemize}

\subsection{Crossover}

Implement one-point or two-point crossover between two selected parents.

\subsection{Mutation}

For each gene in the chromosome, mutate it with a probability $p_m = \frac{1}{\text{number of boxes}}$.

\begin{itemize}
    \item Mutation means reassigning the gene to a new random bin (0 to $N-1$).
    \item Ensure that genes only hold integer values.
\end{itemize}

\section{Q4: GA Execution}

\begin{itemize}
    \item Run the GA for 50 generations with a population size of 20 considering 10, 25, 50, and 100 orders separately.
    \item Vary the number of population size and max generation to see the effect on performance.
    \item Plot the best and average fitness (i.e., number of bins used) per generation.
    \item Report the best solution found and its bin-wise packing configuration.
\end{itemize}

\section{Q5: Analysis and Comparison}

\subsection{Convergence Analysis}

Compare the convergence behavior of different runs (e.g., different random seeds or order sizes).

\subsection{Performance Comparison}

Discuss whether the algorithm tends to find feasible solutions early or late in the process.

\section{Conclusion}
This study successfully implemented...

\end{document}