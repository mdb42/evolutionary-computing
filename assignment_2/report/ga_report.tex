\documentclass[journal,12pt,onecolumn]{IEEEtran}
\usepackage{mathtools,amssymb,amsfonts}
\usepackage{algorithmic}
\usepackage{algorithm}
\usepackage{graphicx}
\usepackage{xcolor}
\usepackage{float}
\usepackage{setspace}
\usepackage{subcaption}
\usepackage[hidelinks]{hyperref}

% Double spacing
\doublespacing

\hypersetup{
   colorlinks=true,
   linkcolor=blue,
   citecolor=black,
   urlcolor=blue
}

% Improve section spacing
\usepackage{titlesec}
\titlespacing*{\section}{0pt}{12pt plus 4pt minus 2pt}{12pt plus 2pt minus 2pt}
\titlespacing*{\subsection}{0pt}{12pt plus 4pt minus 2pt}{8pt plus 2pt minus 2pt}
\titlespacing*{\subsubsection}{0pt}{12pt plus 4pt minus 2pt}{6pt plus 2pt minus 2pt}

\title{Genetic Algorithms: Binary Encoding Optimizing De Jong Test Functions}
\author{
   \IEEEauthorblockN{Matthew D. Branson} \\
   \IEEEauthorblockA{\textit{Department of Computer Science} \\
   \textit{Missouri State University}\\
   Springfield, MO \\
   branson773@live.missouristate.edu
   }
}

\date{June 24, 2025}

\begin{document}

\maketitle

\begin{abstract}
Here is where I write the abstract.
\end{abstract}

\begin{IEEEkeywords}
Genetic Algorithms, Binary Encoding, De Jong Test Functions, Optimization
\end{IEEEkeywords}

\section{Q1: Binary Encoding and Initialization}

\subsection{Variable Encoding}
Here is where I describe the variable encoding using binary strings (8-bit and 16-bit)

\subsection{Generating Initial Population}
Here is where I write about the function to generate a random initial population of binary chromosomes.

\section{Q2: Chromosome Decoding and Function Evaluation}

\subsection{Decoding a Chromosome}
Here is where I write about decoding a binary chromosome into a real number using linear mapping to the domain of each De Jong function.

\subsection{Function Evaluation}
Here is where I write about implementing evaluation functions for all of the De Jong functions.

\section{Q3: Genetic Algorithm Operations}

\subsection{Fitness Proportionate Selection}
Here is where I write about the fitness proportionate selection method using the roulette wheel selection algorithm.

\subsection{Crossover}
Here is where I write about the one-point or two-point crossover operation for binary chromosomes (probability = 0.90)

\subsection{Bitwise Mutation}
Here is where I write about the bitwise mutation operation (probability = 1/Length of chromosome)

\section{Q4: Genetic Algorithm Execution}

\subsection{Execution of the Genetic Algorithm}
Here is where I write about the genetic algorithm execution (50 generations, population size = 20).

\subsection{Fitness Evaluation}
Here is where I include plots for best fitness and average fitness per generation for each function.

\subsection{Best Solution}
Here is where I write about the best solution found and its decoded real values.

\section{Q5: Analysis and Comparison}

\subsection{Convergence Comparison}
Here is where I write about the convergence behavior of all functions.

\subsection{Optimization Analysis}
Here is where I answer the questions: Which function was easiest to optimize? Which was hardest? Why?

\section{Conclusion}
Here is where I write the conclusion.

\end{document}