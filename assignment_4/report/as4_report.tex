\documentclass[journal,12pt,onecolumn]{IEEEtran}
\usepackage{mathtools,amssymb,amsfonts}
\usepackage{algorithmic}
\usepackage{algorithm}
\usepackage{graphicx}
\usepackage{xcolor}
\usepackage{float}
\usepackage{setspace}
\usepackage{subcaption}
\usepackage[hidelinks]{hyperref}
\usepackage{multirow}

\doublespacing

\hypersetup{
   colorlinks=true,
   linkcolor=blue,
   citecolor=black,
   urlcolor=blue
}

\usepackage{titlesec}
\titlespacing*{\section}{0pt}{12pt plus 4pt minus 2pt}{12pt plus 2pt minus 2pt}
\titlespacing*{\subsection}{0pt}{12pt plus 4pt minus 2pt}{8pt plus 2pt minus 2pt}
\titlespacing*{\subsubsection}{0pt}{12pt plus 4pt minus 2pt}{6pt plus 2pt minus 2pt}

\title{Evolutionary Computing: Comprehensive Review}
\author{
   \IEEEauthorblockN{Matthew D. Branson} \\
   \IEEEauthorblockA{\textit{Department of Computer Science} \\
   \textit{Missouri State University}\\
   Springfield, MO \\
   branson773@live.missouristate.edu
   }
}

\date{July 14, 2025}

\begin{document}

\maketitle

\begin{abstract}
This paper presents a comprehensive review of key topics in evolutionary computing, including genetic algorithms, multi-objective optimization, genetic programming, neuroevolution, and co-evolution.
\end{abstract}

\begin{IEEEkeywords}
Genetic algorithms, multi-objective optimization, genetic programming, neuroevolution, NEAT, co-evolution
\end{IEEEkeywords}

% Section A
\section{Genetic Algorithms}

\subsection{Selection Pressure}

Explain the role of selection pressure in Genetic Algorithms. \\
- What are the consequences of too high or too low selection pressure? \\
- Compare roulette wheel selection and tournament selection in this context.

\subsection{Premature Convergence}

Consider a binary-encoded GA solving a minimization problem. \\
- Why might premature convergence occur? \\
- Suggest at least two strategies to mitigate it.

% Section B
\section{Multi-Objective Optimization}


\subsection{Pareto Dominance and Pareto Optimality}

Differentiate between Pareto dominance and Pareto optimality. Illustrate your answer with a two objective optimization example.

\subsection{Diversity Preservation}

Explain the role of diversity preservation mechanisms in MOO.

\subsection{Limitations}

Why can't we scalarize all multi-objective problems into a single objective? Discuss limitations of the weighted sum approach with an example.


% Section C
\section{Genetic Programming}

\subsection{Comparing Genetic Programming and Genetic Algorithms}

Compare Genetic Programming with Genetic Algorithms. \\
- Focus on representation, operators, and typical application domains.

\subsection{Closure and Sufficiency}

Explain how closure and sufficiency properties affect GP design. Give an example of a function and terminal set that satisfies both.

% Section D
\section{Neuroevolution}

\subsection{NeuroEvolution of Augmenting Topologies}

What are the key innovations introduced in NEAT? Explain historical markings, speciation, and complexification.

\subsection{Comparison with Fixed Topology Neuroevolution}

Discuss how NEAT differs from traditional fixed-topology neuroevolution. What advantages does evolving topology offer in dynamic environments?

\subsection{Speciation}

In NEAT, why is speciation important for protecting innovation? How is it implemented, and what are the risks if not used?


% Section E
\section{Co-Evolution}

\subsection{Cooperative vs. Competitive Co-Evolution}
Differentiate between cooperative and competitive co-evolution. Provide an example problem suited for each type.

\subsection{The Red Queen Effect}
Explain the concept of the Red Queen effect in co-evolutionary systems. How can it affect convergence in competitive settings?

\subsection{Challenges in Co-Evolution}
What challenges arise in fitness evaluation in co-evolutionary algorithms? Discuss with respect to relative vs. absolute fitness



\section{Conclusion}

This paper has provided a comprehensive overview of key concepts in evolutionary computing, including genetic algorithms, multi-objective optimization, genetic programming, neuroevolution, and co-evolution.

\end{document}